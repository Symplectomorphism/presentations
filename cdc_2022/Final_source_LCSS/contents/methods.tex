\section{Methods} \label{sec:methods}


The goal of this paper is to systematically design robust controllers for
underactuated systems by combining the universal approximation capability of
neural networks with the intrinsic stabilization properties of PBC.
%
% We propose two learning-based control design frameworks: 1) parameterize the
% controller by the gradient of a neural network and optimize the control
% objective based directly on the evolution of the closed-loop system; and 2)
% solve the matching PDEs~\eqref{eq:pde_main} of IDA-PBC by formulating a neural
% network optimization problem while ensuring the relevant constraints remain
% satisfied. 
%
% We propose two learning-based PBC design frameworks. 
%
In this approach, which we refer to as \textsc{NeuralPbc}, we employ a neural
network $H_d^\theta$ as a surrogate for $H_d$, and the control function is
parametrized by its gradient $\nabla_x H_d^\theta$.
%
The control objective is achieved by optimizing the evolution of the closed-loop
trajectories of~\eqref{eq:hamiltonian_dynamics}.
%
% The second approach (\textsc{NeuralIdaPbc}) leverages IDA-PBC and restricts
% $H_d$ to the form given by~\eqref{eq:desired_hamiltonian}. 
% %
% Here we obtain the controller by numerically solving the
% PDEs~\eqref{eq:pde_main} over a discretization of the state space, with the
% solutions $M_d, V_d, J_2$ approximated by the surrogates $M_d^\theta,
% V_d^\theta, J_2^\theta$.
%


For the \textsc{NeuralPbc} learning problem, we devise two approaches to find a
suitable solution. 
%
The first approach obtains a point estimate of the neural net parameters via
stochastic gradient descent (SGD). We shall refer to these point estimates as
the \textit{deterministic} parameters for the remainder of this document.
%
The second approach obtains a probability distribution for the parameters via
Bayesian inference.
%
The latter addresses concerns about model uncertainties and measurement noise.

\begin{rem}
    We provide a thorough comparison of the deterministic and Bayesian
    solutions to the optimal control of a one-dimensional linear dynamical
    system that is subject to parameter and measurement uncertainties
    in~\cite{ashenafi_robust_control_design}. 
    %
    The simplicity of the underlying system facilitates the computation of
    exact solutions, revealing the fact that the Bayesian solution improves
    the robustness of the closed-loop system against both types of
    uncertainties.
    %
    This comparison serves as a theoretical justification for the
    improved robustness properties of controllers obtained via Bayesian learning
    techniques.

    % We provide a thorough comparison of the deterministic and Bayesian
    % solutions to the optimal control of a one-dimensional linear dynamical
    % system that is subject to parameter and measurement uncertainty in [23].

    % We provide a thorough comparison of the
    % deterministic and Bayesian solutions to the optimal control of a
    % one-dimensional linear dynamical system that is subject to parameter and
    % measurement uncertainty. The simplicity of the underlying system
    % facilitates the computation of exact solutions, revealing the fact that
    % the Bayesian solution improves the robustness of the closed-loop system
    % against parameter and measurement uncertainties.
\end{rem}


% The proposed methods thus far sum up to several possible combinations of
% techniques for finding controllers.
% %
% While any combination can lead to a satisfactory controller, we have found that
% certain combinations complement each other better than others.
% %
% Section~\ref{ssec:comparisons_of_methods} highlights each method's strengths and
% provides some guidelines for choosing the most appropriate combination for
% different use cases.
%
% The advantages of choosing one of these approaches over the other is
% summarized in Table~\ref{sec:comparison_of_methods}.

%%%%%%%%%%%%%%%%%%%%%%%%%%%%%%%%%%%%%%%%%%%%%%%%%%%%%%%%%%%%%%%%%
%%%%%%%%%%%%%%%%%%%%%%%%%%%%%%%%%%%%%%%%%%%%%%%%%%%%%%%%%%%%%%%%%

\subsection{\textsc{NeuralPbc} Problem Statement}
\label{ssec:ml-pbc}
In this framework, rather than limiting $H_d$ to be of a certain form, we
leverage the expressive power of neural networks and formulate an optimization
problem to automatically come up with a suitable function. 
%
We aim to find an energy-like function $H_d$ such that the closed-loop
trajectories converge to the vicinity of the desired equilibrium $x^\star$, at
which point a linear controller, in particular LQR, is employed for
stabilization.
%
% This control objective is encoded in a cost function to be minimized.

Let $H_d^\theta: \mathcal{X} \to \mathbb{R}$ be a neural network, and
$K_v^\theta \in \mathbb{R}^{m \times m}$ be a positive-definite matrix. 
%
Let
$\phi = \phi \left(t; x_0, u \right)$ denote the flow of the equations of
motion~\eqref{eq:hamiltonian_dynamics} with the initial condition $x_0 \in
\mathcal{X}$. 
%
With the decision variables $\theta$ comprising the parameters of $H_d^\theta$
and the entries of $K_v^\theta$, we formulate the minimization problem:
%
\begin{equation}
    \begin{aligned}
        \underset{\theta}{\textrm{minimize}} 
        & & &\int_{0}^{T} \ell \left(\phi,u^{\theta}(\phi) \right) \, \dd t , \\%
        \textrm{subject to}
        & & \dot{x} &= \bmat{\phantom{-}\nabla_p H \\ -\nabla_q H} + \bmat{0 \\ G(q)}u^{\theta}, \\%
        & & u^{\theta} &= -G^{\dagger} \nabla_q H_d^{\theta} - K_v^{\theta} G^\top \nabla_p H_d^{\theta},%
        % & \quad x(0) &= x_0 \in \mathcal{X}, \\%
    \end{aligned}
    \label{eq:neural_pbc_finite_optim}
\end{equation}
%
where $T>0$ is the time horizon, and $\ell: \mathcal{X} \times \mathcal{U}
\rightarrow \mathbb{R}$ is a running cost function to be defined. 
%
% The isolated minimum constraints on $H_d^\theta$ are relaxed by encoding into
% the loss function $\ell$ the task of bringing the states close to the desired
% equilibrium $x^\star$. 
%
Henceforth, we shall refer to each trajectory $\gamma : t \to \phi(t; x_0,
u^\theta)$ as a prediction from the initial state $x_0$ with the current
control law.


The running cost function $\ell$ that yields the objective function
of~\eqref{eq:neural_pbc_finite_optim} consists of the set distance,
$\ell_{\textrm{set}}$, between the current prediction $\gamma$ and the goal set
$\mathcal{S}$ containing a neighborhood of $x^\star$; and the terminal loss,
$\ell_T$, between the final state of the trajectory and $x^\star$. The running
cost is the sum of these components:

%
\begin{equation}
    \ell := \ell_{\textrm{set}}(\gamma) + \ell_T\left(\gamma\right),
    \label{eq:deterministic_loss} 
\end{equation}
%
where each term depends on the decision variables through the
dependence of the system trajectory on $\theta$.


%
% Each component of $\ell$ is defined as follows.
% \begin{enumerate}[topsep=-\parskip+3pt, itemsep=3pt, leftmargin=0pt, itemindent=2.4\listparindent, labelsep=2pt]
%     \item 
    The set distance $\ell_{\textrm{set}}$ is constructed by defining a convex open
    neighborhood $\mathcal{S}$ containing $x^\star$ and computing the set distance of
    $\gamma$ to $\mathcal{S}$:
    %
    % \begin{align}
    %     \ell_{\textrm{set}}(x) & = \textrm{dist}(\gamma, \mathcal{S})\\ 
    %     & = \underset{t}{\inf} \, {\{ \norm{x - y}{}: x \in \gamma(t),\, y \in \mathcal{S} \}}.  
    %     \label{eq:set_distance}
    % \end{align}
    \begin{equation}
        \ell_{\textrm{set}}(\gamma)
        = \underset{t}{\inf} \, {\{ \norm{a - b}{}: a \in \gamma(t),\, b \in \mathcal{S} \}}.  
        \label{eq:set_distance}
    \end{equation}
    % 
    For instance, the set $\mathcal{S}$ may be chosen as a ball around of radius
    $r$ around $x^\star$ in the standard norm topology. 
    %
    Here, $r$ becomes a hyperparameter of the training algorithm. With a
    particular choice of $\mathcal{S}$, if at any point along the prediction
    $\gamma$, a state $x$ is closer than $r$ to $x^\star$, no penalty is
    incurred by $\ell_{\textrm{set}}$. 
    %
    Moreover, we introduce a terminal loss of the form $\ell_T= \vectornorm{x(T)
    - x^\star}$. 
    %
    While set distance loss encourages trajectories to approach the neighborhood
    of $x^\star$ at any point along the prediction, the terminal loss encourages
    trajectories to remain as close to the desired state as possible at
    time $T$.
    %
%     \item \textit{The transversal distance} $\ell_{\bot}$ to a preferred orbit $\gamma^\star$
%     is defined in terms of the transverse
%     coordinates~\cite{manchester2011transverse}, denoted by $x_\bot$, along
%     $\gamma^\star$. 
%     %
%     The preferred orbit may be generated using any motion planning algorithm
%     that runs fast enough, such as RRT~\cite{karaman2011anytime},
%     $\textrm{A}^\star$~\cite{choset2005principles}, trajectory
%     optimization~\cite{conway2002practical}, virtual holonomic
%     constraints~\cite{shiriaev2005constructive}, etc. 
%     %
%     In some cases, there are natural choices for these preferred orbits, such as
%     the homoclinic orbit of the pendulum~\cite{spong2001nonlinear}.

%     By construction of the transverse coordinates $x_\bot$, the prediction
%     $\gamma$ converges to $\gamma^\star$ if and only if $x_\bot \to 0$ as $t \to
%     \infty$.
%     %
%     This provides an efficient way to compute the penalty $\ell_{\bot}$ as
%     %
%     \begin{equation}
%         \ell_\bot(x, u) =  x_{\bot}^{\top} Q x_{\bot} + u^{\top}Ru,
%         \; Q \succeq 0, \; R \succeq 0.
%         \label{eq:xperploss}   
%     \end{equation}
%     %
%     If there is no preferable $\gamma^\star$ available, we simply let $Q = 0$
%     and only incur penalty due to the expenditure of control effort $u$.
% \end{enumerate}

\subsection{Sampling of the Initial States} 
\label{sssec:ml-pbc-sampling}

The objective function of~\eqref{eq:neural_pbc_finite_optim} depends on the
initial state $x_0 \in \mathcal{X}$ through the prediction $\gamma : t \to \phi(t; x_0, u)$.
%
We are interested in obtaining a controller that performs well for
all initial states rather than for a single point. 
%
Viewing $x_0$ as a random variable over the state space $\mathcal{X}$, the
objective function of~\eqref{eq:neural_pbc_finite_optim} may be expressed as the
expectation of the loss over the distribution of $x_0$:
%
\begin{equation}
    J(\theta) = \mathbb{E}_{x_0 \sim p(\bx_0)} \Bigl[ \ell(\phi(t;x_0), u) \Bigr].
    \label{eq:loss_expectation}
\end{equation}
%
% In this subsection, we describe a strategy that samples $\mathcal{X}$ efficiently.
%
Instead of randomly sampling the entire space, we adopt the state sampling
techniques from~\cite{ross2011no}.
%
Let $\mathcal{D}$ denote the training dataset of size $N_\mathcal{D}$. 
%
We randomly alternate between populating $\mathcal{D}$ using (a) the
\textsc{DAgger} strategy~\cite{ross2011no}, and (b) sampling from $x_0 \sim
\mathcal{N} \left(x^\star, \Sigma\right)$, where the variance $\Sigma$ is kept
small. 
%
Sampling initial states from $x_0 \sim \mathcal{N} \left(x^\star, \Sigma\right)$
helps recover from locally optimal solutions obtained from trajectories that
predominantly miss the goal set $\mathcal{S}$ by a large amount. 
%
The \textsc{DAgger} strategy samples a collection of initial states $\{x_0\}$
from trajectories $\{\gamma\}$ generated by executing the current policy
$u^\theta$.
%
This technique helps to efficiently sample the relevant part of the state space.
%
After each iteration through the entire dataset, we substitute $N_{\textrm{R}}$
points in $\mathcal{D}$ with new samples, keeping an $N_\mathcal{D} -
N_\textrm{R}$ portion of initial states as the \textit{replay buffer}, a common
technique used in reinforcement learning algorithms~\cite{lin1992self}.
%
% \textsc{DAgger} populates $\mathcal{D}$ with points along the trajectories
% generated by executing the control law $u^\theta$ with the current parameters
% $\theta$.
%
% This iteratively collects new samples from the regions of the state-space
% visited by the learned policy.
%
% This technique helps recover from locally optimal solutions obtained from
% trajectories that predominantly miss the goal set $\mathcal{S}$ by a large
% amount.
%
% Initially, $u^\theta$ is poorly trained, and $\mathcal{D}$ may be dominated by
% points that are far away from the goal set $\mathcal{S}$.
% %
% In high-dimensional state spaces with complicated dynamics, it may be difficult
% to recover from locally optimal solutions if $\{\gamma\}$ predominantly miss the
% goal set $\mathcal{S}$ by a large amount. 
% %
% To circumvent this problem, the training algorithm randomly alternates between
% populating $\mathcal{D}$ using (a) the \textsc{DAgger} strategy, and (b)
% sampling from $x_0 \sim \mathcal{N} \left(x^\star, \Sigma\right)$, where the
% variance $\Sigma$ is kept small. 
% %
% The likelihood of populating $\mathcal{D}$ by
% \textsc{DAgger} is gradually increased as training progresses. 
% %
% This kickstarts the algorithm with a dataset that has several initial states
% already close to the goal set $\mathcal{S}$.
% %
% Finally, after each iteration through the entire dataset, we
% substitute $N_{\textrm{R}}$ points in $\mathcal{D}$ with new samples, keeping a
% $N_\mathcal{D} - N_\textrm{R}$ portion of initial states as the \textit{replay
% buffer}, a common technique used in reinforcement learning algorithms~\cite{lin1992self}.

% The complete details of this technique can be found in~\cite{datadriven_ESC}.


%%%%%%%%%%%%%%%%%%%%%%%%%%%%%%%%%%%%%%%%%%%%%%%%%%%%%%%%%%%%%%%%%
%%%%%%%%%%%%%%%%%%%%%%%%%%%%%%%%%%%%%%%%%%%%%%%%%%%%%%%%%%%%%%%%%

\subsection{Deterministic Solution for \textsc{NeuralPbc}} 
\label{sssec:ml-pbc-deterministic}

A solution to the nonlinear program~\eqref{eq:neural_pbc_finite_optim} can be
obtained using a gradient-based search. 
%
In this work, we use the stochastic gradient descent (SGD) method, which is
standard in machine learning applications. 
%
% We briefly mention here the approach for computing the relevant gradient
% $\nicefrac{\partial J}{\partial \theta}$, which depends on the solution
% $\phi(t;x_0;u^\theta)$ of the ordinary differential equation (ODE) in
% Eq.~\eqref{eq:hamiltonian_dynamics}. 
%
We invoke the adjoint method~\cite{chen2018neural} to compute
$\nicefrac{\partial J}{\partial \theta}$, which depends on the solution
$\phi(t;x_0;u^\theta)$ of the ordinary differential equation (ODE)
provided in~\eqref{eq:hamiltonian_dynamics}. 
%
% The Neural ODE paper~\cite{chen2018neural}, invokes the adjoint method, which is
% based on Pontryagin's maximum principle~\cite{pontryagin1962mathematical}, for
% computing this gradient.
% %
% In~\cite{DBLP:journals/corr/abs-2001-04385}, several automatic differentiation (AD) systems and
% the adjoint method are generalized to enable efficient gradient computation
% through solutions of ODEs. 
% %
% We leverage this recent development to compute the gradient of $J$ and
% iteratively update $\theta$ according to any SGD algorithm of choice to reach a
% satisfactory solution to the optimization
% problem~\eqref{eq:neural_pbc_finite_optim}.
%
Algorithm~\ref{algo:neuralpbc} provides a summary of our training process.

\begin{algorithm}
\caption{Solution to Nonlinear Program~\eqref{eq:neural_pbc_finite_optim}}
\label{algo:neuralpbc}
\small
\begin{algorithmic}[1]
    \algrenewcommand\algorithmicindent{0em} % No indent
    \Procedure{NeuralPBC}{$H,G,\theta$}
    \State $f \,\gets f(H,G)$ dynamics given by ODE~\eqref{eq:hamiltonian_dynamics} with $u = u^\theta$
    \State $\mathcal{D} \gets \{x_0\}_{(N_{\mathcal{D}})}$  \Comment{$N_{\mathcal{D}}$ samples of $x_0$ (Section~\ref{sssec:ml-pbc-sampling})}
    \algrenewcommand\algorithmicindent{1.5em} % Change indent back to default
    \While{$i < $ \texttt{maxiters}}
    \For{each \texttt{batch} $\subset \mathcal {D}$}
        \State $J \gets 0$\Comment{Batch loss}
        \For{each $x_0 \in$ \texttt{batch}}
            \State $\gamma \gets$ integrate $f$ from $x_0$ with current $\theta$
            \State $J \gets J + \ell(\gamma; \theta)$\Comment{Eq.~\eqref{eq:deterministic_loss}}
        \EndFor
        \State $\theta \gets \theta + \alpha_i \nicefrac{\partial J}{\partial \theta}$\Comment{SGD step}
    \EndFor
    \State $\mathcal{D} \gets \{\mathcal{D}\}_{(1,\ldots,N_{\mathcal{D}}-N_{\textrm{R}})} \cup \{x_0\}_{(N_{\textrm{R}})}$\Comment{Replay buffer}
    \State $i \;\:\gets i + 1$
    \EndWhile
    \State \textbf{return} $\theta$
    \EndProcedure
\end{algorithmic}
\end{algorithm}

%%%%%%%%%%%%%%%%%%%%%%%%%%%%%%%%%%%%%%%%%%%%%%%%%%%%%%%%%%%%%%%%%
%%%%%%%%%%%%%%%%%%%%%%%%%%%%%%%%%%%%%%%%%%%%%%%%%%%%%%%%%%%%%%%%%

\subsection{Bayesian Solution for \textsc{NeuralPbc}}
\label{sssec:ml-pbc-bayes}
In this subsection, we formulate the Bayesian learning framework that minimizes
the effects of system parameter uncertainties and measurement errors. 
%
In this framework, the neural net parameterization of $H^\theta_d$ given in
Section~\ref{sssec:ml-pbc-deterministic} is replaced by a Bayesian neural
network whose parameters are uncorrelated random variables sampled from a
multivariate probability distribution (MPD).
%
The goal is to learn the distribution parameters $z$ of the posterior MPD
$q(\theta;z)$ that maximize the \textsc{Elbo} given in~\eqref{eq:elbo}. 
%
% For continuous probability distributions, ELBO is redefined in terms of
% continuous entropy as
% %
% \begin{equation}
%     \mathcal{L}(x,z) = \mathbb{E}_{\theta \sim q} \left[\log(f(x|\theta;z)q(\theta)) - \log(q(\theta)) \right],
%     \label{eqn:cont_entropy_elbo}
% \end{equation}
%
The computation of the \textsc{Elbo} requires the likelihood function and the
prior distribution.
%
The likelihood function is chosen to be a Gaussian probability
distribution of the form
\begin{equation}
    p(J \mid \theta) = \mathcal{N}\left(0, s \right),
    % \text{likelihood} = \frac{1}{\sqrt{2\pi}s}\exp{\left(-\frac{1}{2} \left ( \frac{-\ell(\gamma, u)}{s}\right)^2 \right)}
    \label{eqn:likelihood_neuralpbc}
\end{equation}
where $s$ is the standard deviation of the likelihood and a hyperparameter. 
%
With the choice of the likelihood given in \eqref{eqn:likelihood_neuralpbc},
maximizing the \textsc{Elbo} pulls the loss $J(\theta)$ to zero.
%
% On top of the neural net parameters $\theta$, we wish to learn the posterior
% over the standard deviation $s$ of the likelihood. 
% %
% Hence, the posterior distribution $q(\xi)$ is an MPD over the multivariate
% random variable $\xi$, where $\xi = [\theta, s]$ and $z$ is the distribution
% parameters over $\xi$.
% %
% \begin{rem}
%     For continuous posterior distribution, the \textsc{Elbo} given in
%     equation~\eqref{eq:elbo} is redefined using differential entropy, which
%     expresses the prior and posterior in terms of their probability density
%     functions (PDF). In this case, the likelihood
%     in~\eqref{eqn:likelihood_neuralpbc} is also a PDF and the \textsc{Elbo} is
%     not bounded by zero.
% \end{rem}

% We take two approaches to selecting the distribution parameters $z_0$ of the
% prior. The simplest approach is to use an uninformed prior with randomly
% initialized $z_0$ to start the optimization problem; this choice encourages
% exploration but has slow convergence properties. The second approach uses an
% informed prior that warm-starts the Bayesian training around the solution of the
% deterministic training. To do so, $z_0$ is selected such that the prior
% distribution is centered around the parameters learned from the deterministic
% technique discussed in Section~\ref{sssec:ml-pbc-deterministic}.


System parameter uncertainties and measurement noise can deteriorate the
performance of controllers when employed on real systems. 
%
Hence, in the Bayesian framework, we inject these uncertainties directly into
the training loop in order to learn a controller that works for a wide range of
system parameters and measurement noise.
%
We model system parameter uncertainties by sampling a set of system parameters
$\zeta$ from a normal distribution $\mathcal{N}(\zeta_0, \sigma_{\zeta})$
centered around a nominal parameter $\zeta_0$.
%
Each time we generate a new trajectory starting from the initial states drawn
from the replay buffer $\mathcal{D}$, we draw a new sample of $\zeta$.
%


Additionally, we model measurement error by injecting noise into the prediction
$\gamma$.
%
This is achieved by replacing the dynamical constraint given
in~\eqref{eq:neural_pbc_finite_optim} with the following stochastic differential
equation (SDE).
\begin{align}
    \dd x &= \left(\bmat{\phantom{-}\nabla_p H \\-\nabla_q H} + \bmat{0 \\ G(q)}u^{\theta}(x) \right) \dd t + \nabla_x u(x) \dd W_t, 
    \label{eq:sde_initial}
\end{align}
where $\dd W_t$ is a correlated noise process, such as Wiener process, on the
states due to measurement uncertainties, and $\nabla_x u(x)$ is the coefficient
of the first-order Taylor approximation of $u^{\theta}(x)$ around zero noise. 
  
\begin{rem}
    Introducing uncertainties to the deterministic training finds a point
    estimate of the optimal controller parameter, which may be interpreted
    as the mean of the optimal posterior distribution that Bayesian learning
    provides. A point estimate of the learned parameters is prone to be
    biased (for example, if the uncertainty in system parameters is large,
    the optimal parameter $\theta$ for the true system parameter may be
    quite far from the deterministic solution). This bias-variance trade-off
    problem is alleviated by Bayesian inference which allows one to
    marginalize over the posterior parameter
    distribution~\cite{bishop2006pattern}.
\end{rem}  


%
This Bayesian framework extends the deterministic training process discussed in
Section~\ref{sssec:ml-pbc-deterministic}, by replacing lines 7 to 10 of
Algorithm~\ref{algo:neuralpbc} with Algorithm~\ref{algo:neural_bayes_pbc}.
%
% Similar to the deterministic training, the SGD update shown in line 8 of
% Algorithm~\ref{algo:neural_bayes_pbc} can be replaced by more refined variants
% such as \textsc{ADAM}, \textsc{ADAGrad}~\cite{lydia2019adagrad}, etc. 
%
% The computation of the gradient ${\partial \mathcal{L}}/{\partial z}$ comes with
% two complications. First, the gradient requires differentiating the
% \textsc{Elbo} through the solution $\phi(t;x_0;u^\theta)$ of an ODE. We address
% this issue by using the adjoint method briefly discussed in
% Section~\ref{sssec:ml-pbc-deterministic}. 
%
Similar to the deterministic training, we use the adjoint method to compute the
gradient ${\partial \mathcal{L}}/{\partial z}$ through the solution of the SDE
given in~\eqref{eq:sde_initial}.
%
Moreover, we invoke the reparameterization trick of the Automatic
Differentiation Variational
Inference(\textsc{ADVI})~\cite{kucukelbir2015automatic} to compute the gradient
of samples $\theta$ with respect to the distribution parameters $z$.
%
% Second, computing ${\partial \mathcal{L}}/{\partial z}$ requires the derivative
% of the sample $\xi$ with respect to the distribution parameters $z$, which is
% intractable. We handle this complication by invoking the reparameterization
% trick of the Automatic Differentiation Variational
% Inference(\textsc{ADVI})~\cite{kucukelbir2015automatic}.



\begin{algorithm}
    \caption{Bayesian Learning}\label{algo:neural_bayes_pbc}
    \small
    \begin{algorithmic}[1]
        % \algrenewcommand\algorithmicindent{0em} % No indent
        % \State Select inform or uninformed prior $z$
        \State $f \,\gets f(H,G)$ dynamics given by SDE~\eqref{eq:sde_initial} with $u = u^\theta$
        \For{each $x_0 \in$ \texttt{batch}}
        \State Initialize $\mathcal{L} = 0$
        % \algrenewcommand\algorithmicindent{1.5em} % Change indent back to default
        \For{i=1:N}
            \State $\theta, s \sim q(\theta; z)$ \Comment{Sample parameters of $H^\theta_d$}
            \State $\zeta \sim \mathcal{N}(\zeta_0,\sigma_{\zeta})$ \Comment{Sample system parameters}
            \State $\gamma \gets$ integrate $f$ from $x_0$ 
            \State $p(J, \theta) = \frac{1}{s\sqrt{2\pi}}\exp{\left(-\frac{1}{2} \left ( \frac{J(\theta)}{s}\right)^2 \right)} p(\theta)$
            \State $\mathcal{L} \gets \mathcal{L} + \frac{1}{N}\left (\log[p(\ell, \theta)] - \log[q(\theta;z)]\right)$
        \EndFor
        \EndFor
        \State $z \gets z + \alpha_i \left({\partial \mathcal{L}}/{\partial z}\right)$
    \end{algorithmic}
\end{algorithm}

%%%%%%%%%%%%%%%%%%%%%%%%%%%%%%%%%%%%%%%%%%%%%%%%%%%%%%%%%%%%%%%%%
%%%%%%%%%%%%%%%%%%%%%%%%%%%%%%%%%%%%%%%%%%%%%%%%%%%%%%%%%%%%%%%%%
