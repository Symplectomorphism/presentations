\section{Introduction}


\IEEEPARstart{P}{assivity}-based control (PBC)~\cite{ortega2001putting,van2000l2} is a
methodology that casts the control problem as a search for an interconnection
pattern among subsystems such that the overall dynamics exhibits passivity
properties, which help infer stability.
%
The standard PBC approach in the robot control domain amounts to shaping the
system's overall energy with a fictitious one and injecting damping to
achieve asymptotic stability.
%
In practice, the closed-loop energy function is typically chosen as a quadratic
function of the error to the desired equilibrium point.
%
However, a limit of this approach is the lack of a clear connection between the
form of the desired energy function and the performance objective of a given
task.
%
Furthermore, control synthesis in PBC methods relies strongly on the dynamical
model, and imperfect compensation of the system's energy may lead to poor
performance or even instability.
%
This raises concerns regarding model uncertainties, especially in dynamical
systems with large number of parameters~\cite{nagy,wu2020active}.
%
% Although passive systems admit some inherent robustness
% properties~\cite{van2000l2}, imperfect compensation of the system's energy in
% PBC may result in instability.
%


There are techniques that combine tools from optimization, probability theory,
and machine learning to learn control strategies from inaccurate system models
or even unknown dynamics.
%
% Researchers have been tackling similar problems using a combination of tools
% from optimization, probability theory, and machine learning.
%
% Reinforcement learning (RL)~\cite{sutton2018reinforcement} uses data-driven
% techniques to find approximate solutions to the Hamilton-Jacobi-Bellman
% equation, which is the quintessential task of optimal control.
%
One example in this domain is reinforcement learning (RL), wherein the quintessential
task is to find a mapping from the observed states to the control input such
that the long-term cumulative reward is maximized.
%
Many variants of RL view the dynamical system as a black-box, which may be
represented by a real system or a simulation, and rely on repeated interactions
with the unknown environment to improve the control law.
%
One way to account for model uncertainties within this method is to perform training
directly on the real system.
%
% However, to avoid damage, training may require conservative state constraints,
% consequently limiting the capabilities of the learning
% techniques~\cite{ARGALL2009469,beaudoin2021structured}.
%
While RL methods evidently offer more flexibility on how they learn from the
unknown environment~\cite{heess2017emergence, andrychowicz2020learning,
lillicrap2015continuous}, most of them discard any potential geometric or
algebraic structures that may be useful in control synthesis.
%
This prohibitively increases the sample complexity and limits the ability to
infer stability properties.


% In contrast to black-box approaches, constructing learning frameworks based on
% prior knowledge and physical principles have been shown to be much more
% efficient~\cite{zhong2019symplectic,greydanus2019hamiltonian,sirichotiyakul2020data}.
% %
% In~\cite{zhong2019symplectic}, a learning framework for discovering Hamiltonian
% dynamics is proposed, and the controller is synthesized using passivity-based
% techniques.
% %
% We proposed in~\cite{sirichotiyakul2020data} a data-driven PBC framework where
% the desired energy function is expressed by a neural network, which is optimized
% based directly on the performance of the closed-loop trajectories of the system.
% %
% A similar approach is again explored in~\cite{massaroli2021optimal}.
% %
% However, these model-based approaches do not present a trivial solution to
% address uncertainties in the nominal dynamical model from which they are
% formulated.


Bayesian learning~\cite{gal2016improving,thakur} offers an alternative method to
simultaneously combat model uncertainties while preserving the useful physical
structure in a data-driven framework.
%
% It has already proven effective in the modeling of dynamical systems. 
%
For instance, it is used to model disturbances, such as the effect of wind gusts
on quadcopters and the motion of other vehicles in autonomous
driving~\cite{sadigh2015safe}. 
%
% Nonlinear dynamics can be decomposed into segments of linear dynamics, and learn
% the linear dynamical units and their transition probabilities through
% BL~\cite{pmlr-v54-linderman17a}. 
%
% Bayesian learning is also used in motion planning and control synthesis. 
%
In human-robot interactions, such as prosthetics and rehabilitation devices,
Bayesian neural networks can classify and predict motions in order to generate
reliable commands for the device~\cite{motionclassification}. 
% %
% Given an optimal policy, \cite{thakur} uses Bayesian neural networks to detect
% uncertainties caused by offset between training and testing conditions. When the
% uncertainty is high, the controller switches to a fallback strategy that
% requests more demonstrations. 
%
% In~\cite{lazkano2007use}, a Bayesian neural network trained using data generated
% only from sonar readings is used to generate task-execution commands for
% mobile robots.
%
% A unified learning-based approach that incorporates combines traditional control
% theory and addresses uncertainties is still missing from the literature.
%
However, the literature on applications of Bayesian learning in the control of
uncertain dynamical systems is scarce.

% {
%     \color{magenta}
%     We present a theoretical justification for the robustness properties of
%     controllers obtained from Bayesian learning techniques in~[23]. In [23],
%     we compare the performance of deterministic and Bayesian solutions to
%     the optimal control search problem for a simple dynamical system under
%     uncertainties. Our analysis shows that the deterministic solution is
%     sensitive to system parameter uncertainty and measurement noise.
%     Moreover, the Bayesian solution reasons about the stability of the
%     system under these uncertainties. In the following sections, we extend
%     the findings of [23] to a more complex control synthesis technique, such
%     as PBC.
    
% } 


In this work, we present a unified framework that simultaneously combines PBC
techniques and rigorously addresses model uncertainties using Bayesian learning.
%
The specific contributions of this work are threefold: (i)
Motivated by~\cite{sirichotiyakul2020data}, we incorporate uncertainties into
the dynamics and cast the passivity-based control synthesis problem as a
stochastic optimization problem.
%
The closed-loop storage (energy-like) function, from which the control law is
derived, is not restricted to a certain form and instead represented by a neural
network whose parameters are random variables.
%
% The second aim is to develop an algorithm that find a suitable family of
% parameters for the energy-like function such that the behavior of the
% closed-loop trajectories optimizes a certain performance objective.
%
(ii) We apply Bayesian learning and develop an algorithm that finds
a suitable probability distribution of the neural net parameters automatically.
%
In contrast to deterministic optimization, this approach provides a probability
distribution over the neural net parameters instead of a point estimate,
providing a way to reason about model uncertainties and measurement noise during
the learning process.
%
(iii) We demonstrate the efficacy and robustness of our current framework with a
comparison against our deterministic
framework~\cite{sirichotiyakul2020data}. The comparison is performed on a
benchmark underactuated control problem--the inertia wheel pendulum--both in
simulation and real-world experiments. 



% {
 %   \color{magenta}
    %
    % The contributions of this work are summarized as follows:
    % \begin{enumerate}
    %     % \item Without destroying the structure of the underlying physical system, we
    %     % leverage the intrinsic stability properties of the IDA-PBC method and
    %     % formulate an optimization problem that finds surrogates for the solutions to
    %     % the essential nonlinear PDEs.
    %     \item Formulate a stochastic optimization problem that addresses system parameter and measurement uncertainties and preserves the properties of PBC presented in [12],
    %     \item Solve the stochastic optimization problem via Bayesian learning techniques,
    %     \item Compare the robustness properties of the solution obtained from Bayesian learning techniques with the deterministic PBC solution from [12].
    % \end{enumerate} 
% }
% The summary of our contributions is as follows:
% %
% \begin{itemize}[label=\diamond,topsep=-\parskip+3pt, itemsep=0pt, leftmargin=0pt, itemindent=2.4\listparindent, labelsep=2pt]
%     \item \textsc{NeuralPbc}, which finds the desired energy-like
%     function that optimizes the behavior of the closed-loop trajectories
%     \item A Bayesian learning framework, which infers robustness of
%     \textsc{NeuralPbc} against model uncertainties.
%     \item Validation of our methods on the inertia wheel pendulum, a benchmark
%     underactuated control problem, both in simulation and on a real system.
% \end{itemize}



% In order to learn a robust controller in simulation, we modify our deterministic
% learning technique into a Bayesian framework. 
% %
% The development of controllers under the Bayesian framework has two main
% advantages. 
% %
% First, Bayesian learning can account for uncertainties by learning a
% probability distribution over the parameters of the controller. 
% %
% Hence, we leverage this technique to minimize the adverse effects of system
% parameter uncertainties and measurement error on the control parameters.
% % First, any deterministic data-driven learning technique can be modified into a
% % Bayesian learning framework that accounts for uncertainties by learning a
% % probability distribution over the parameters of the controller. 
% %
% % Second, Bayesian learning can improve on parameters learned from deterministic
% % learning techniques in simulation to guarantee robust performance on the real
% % system. 
% %
% Second, we can use parameters learned from deterministic learning techniques as
% priors to the Bayesian learning. 
% %
% This allows us to improve the robustness properties of the deterministic
% training without learning from scratch.
% %
% To best demonstrate these concepts, we use data-driven passivity-based-control
% (PBC) as the deterministic learning technique on which we improve.

% % This paper has two main contributions: 1) deterministic data-driven PBC
% % techniques are used to learn energy-like function of the closed-loop system
% % parameterized by a neural network, 2) a Bayesian learning framework is injected
% % into the deterministic learning scheme to learn a controller robust against
% % uncertainties created by an offset between simulation training and experimental
% % testing. 
% % %
% % The remainder of the paper is organized as follows: {\todo[inline]{ complete
% % once all sections are known}}

